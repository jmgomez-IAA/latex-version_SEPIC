
\section {Prologo}
\subsection {Logistica inicial}
\begin{frame}
  \frametitle {Logistica inicial}

  \begin{alertblock}{Wifi del Curso}
    SSID: sepic

    Password: ps2N6zTu
  \end {alertblock}
  
\end{frame}


\begin {frame}
  \frametitle {Objetivos}
  \begin{itemize}
  \item Conocer las caracter�sticas b�sicas de los sistemas embebidos para instrumentaci�n.
  \item Conocer las nuevas tendencias de desarrollo en el dominio de los sistemas en un chip.
  \item Manejar herramientas de desarrollo para construir sistemas embebidos.
  \item Configurar un sistema embebido basado en microprocesador y Sistema Operativo.
  \item Conocer y manipular los entornos de programaci�n C y Python.
  \item Usar las capacidades de comunicaci�n m�s habituales de un sistema embebido.
  \item Crear conexiones con elementos electr�nicos como leds, rel�s, sensores, etc.
  \item Controlar los puertos de entrada y salida digitales de un sistema embebido.
  \item Dise�ar y adaptar un sistema embebido para instrumentaci�n cient�fica.
  \end{itemize}
\end {frame}


\begin{frame}
  \frametitle {Teor�a}
  \begin{enumerate}[I]
  \item Introducci�n.
    \begin{itemize}

    \item  Definici�n de sistemas embebidos.
    \item  Microprocesadores, microcontroladores y dispositivos l�gicos programables.
    \item Software embebido y Sistemas Operativos.
    \item Plataformas libres: Arduino y Raspberry Pi.
    \end {itemize}
  \item Sistema embebido de referencia: Raspberry Pi.
    \begin{itemize}
    \item Versiones y caracter�sticas.
    \item Sistemas Operativos disponibles.
    \item Entornos de programaci�n en la Raspberry Pi: C y Python.
    \item Comunicaciones digitales: tipos de comunicaciones y librer�as.
    \item Actuadores y sensores m�s habituales.
    \end{itemize}
  \end{enumerate}
\end{frame}

